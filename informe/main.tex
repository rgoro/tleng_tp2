\documentclass{article}
\usepackage[utf8]{inputenc}

\title{Informe TL}
\date{Julio 2015}

\usepackage{natbib}
\usepackage{graphicx}
\usepackage{caratula}
\usepackage[spanish]{babel}

\begin{document}

\newcommand{\num}{\textit{\textbf{num}}}
\newcommand{\var}{\textit{\textbf{constante}}}


%%%%%%%%%%%%%%%%%%%%%%%%%%%
%			INICIO DE CARÁTULA			%
%%%%%%%%%%%%%%%%%%%%%%%%%%%

%% **************************************************************************
%
%  Package 'caratula', version 0.2 (para componer caratulas de TPs del DC).
%
%  En caso de dudas, problemas o sugerencias sobre este package escribir a
%  Nico Rosner (nrosner arroba dc.uba.ar).
%
% **************************************************************************



% ----- Informacion sobre el package para el sistema -----------------------

\NeedsTeXFormat{LaTeX2e}
\ProvidesPackage{caratula}[2003/4/13 v0.1 Para componer caratulas de TPs del DC]


% ----- Imprimir un mensajito al procesar un .tex que use este package -----

\typeout{Cargando package 'caratula' v0.2 (21/4/2003)}


% ----- Algunas variables --------------------------------------------------

\let\Materia\relax
\let\Submateria\relax
\let\Titulo\relax
\let\Subtitulo\relax
\let\Grupo\relax


% ----- Comandos para que el usuario defina las variables ------------------

\def\materia#1{\def\Materia{#1}}
\def\submateria#1{\def\Submateria{#1}}
\def\titulo#1{\def\Titulo{#1}}
\def\subtitulo#1{\def\Subtitulo{#1}}
\def\grupo#1{\def\Grupo{#1}}


% ----- Token list para los integrantes ------------------------------------

\newtoks\intlist\intlist={}


% ----- Comando para que el usuario agregue integrantes

\def\integrante#1#2#3{\intlist=\expandafter{\the\intlist
	\rule{0pt}{1.2em}#1&#2&\tt #3\\[0.2em]}}


% ----- Macro para generar la tabla de integrantes -------------------------

\def\tablaints{%
	\begin{tabular}{|l@{\hspace{4ex}}c@{\hspace{4ex}}l|}
		\hline
		\rule{0pt}{1.2em}Integrante & LU & Correo electr\'onico\\[0.2em]
		\hline
		\the\intlist
		\hline
	\end{tabular}}


% ----- Codigo para manejo de errores --------------------------------------

\def\se{\let\ifsetuperror\iftrue}
\def\ifsetuperror{%
	\let\ifsetuperror\iffalse
	\ifx\Materia\relax\se\errhelp={Te olvidaste de proveer una \materia{}.}\fi
	\ifx\Titulo\relax\se\errhelp={Te olvidaste de proveer un \titulo{}.}\fi
	\edef\mlist{\the\intlist}\ifx\mlist\empty\se%
	\errhelp={Tenes que proveer al menos un \integrante{nombre}{lu}{email}.}\fi
	\expandafter\ifsetuperror}


% ----- Reemplazamos el comando \maketitle de LaTeX con el nuestro ---------

\def\maketitle{%
	\ifsetuperror\errmessage{Faltan datos de la caratula! Ingresar 'h' para mas informacion.}\fi
	\thispagestyle{empty}
	\begin{center}
	\vspace*{\stretch{2}}
	{\LARGE\textbf{\Materia}}\\[1em]
	\ifx\Submateria\relax\else{\Large \Submateria}\\[0.5em]\fi
	\par\vspace{\stretch{1}}
	{\large Departamento de Computaci\'on}\\[0.5em]
	{\large Facultad de Ciencias Exactas y Naturales}\\[0.5em]
	{\large Universidad de Buenos Aires}
	\par\vspace{\stretch{3}}
	{\Large \textbf{\Titulo}}\\[0.8em]
	{\Large \Subtitulo}
	\par\vspace{\stretch{3}}
	\ifx\Grupo\relax\else\textbf{\Grupo}\par\bigskip\fi
	\tablaints
	\end{center}
	\vspace*{\stretch{3}}
	\newpage}




\materia{Teor\'ia de Lenguajes}
\submateria{Primer Cuatrimestre de 2015}
\titulo{Trabajo Práctico 2}

\grupo{Grupo Estado Final}
\integrante{Gorojovsky, Román}{530/02}{rgorojovsky@gmail.com}
\integrante{Lazzaro, Leonardo}{147/05}{lazzaroleonardo@gmail.com}

\begin{titlepage}
\maketitle
\thispagestyle{empty}
\end{titlepage} 

%%%%%%%%%%%%%%%%%%%%%%%%%%%
%				FIN DE CARÁTULA			%
%%%%%%%%%%%%%%%%%%%%%%%%%%%

\section*{Introducción}
El trabajo consiste en implementar un programa que tome un archivo escrito en el lenguaje Musileng,
definido por la cátedra, y convertirlo en un archivo midi, pasando por un lenguaje intermedio que es
interpretado por el programa \emph{midicomp} para generar finalmente el archivo midi.

El problema, entonces puede dividirse en tres subproblemas: 

\begin{itemize}
	\item Convertir la descripción del lenguaje Musileng en una gramática bien definida.
	\item Aprender a usar alguna biblioteca preexistente para convertir esa gramática en código
	\item Convertir la salida del \emph{parser} creado en los dos pasos anteriores en el lenguaje de
		\emph{midicomp}
\end{itemize}

\section*{Decisiones tomadas}
Eligimos usar \emph{PLY} debido a nuestro mejor manejo del lenguaje \emph{Python} y usamos como
esqueleto del trabajo el código presentado en clase:

\begin{itemize}
	\item \texttt{lexer\_rules.py} contiene las definiciones y reglas de tokens
	\item \texttt{lexer.py} es el código de ejecución del \emph{lexer} (usado principalmente para
		testeo.
	\item \texttt{parser\_rules} contiene la definición de la gramática
	\item \texttt{expressions.py} contiene los objetos que se crean a partir del análisis sintáctico.
	\item \texttt{parser.py} es el código de ejecución del \emph{parser}
\end{itemize}

Como se verá en la sección que detalla la implementación, la conversión de objetos al lenguaje
\emph{midicomp} está implementada dentro de los objetos de \texttt{expressions.py}, usando patrones
de programación orientada a objetos.

\pagebreak
\section*{Gramática}
Definimos la siguiente gramática, para cuya definición priorizamos la simplicidad en la construcción
de los objetos que luego se usarán para generar el archivo \emph{midicomp} por sobre la legibilidad
de la gramática. En general pusimos los terminales (puntos y coma, llaves de cierre) en las
producciones ``de más afuera''.

Notamos en negrita los tokens y en negrita bastardilla los tokens con algún valor,
definidos más abajo.
$\\$

\fbox{
	\begin{minipage}{33em}
		Musileng $\rightarrow$ DefTempo DefCompas Constantes Voces\\
		DefTempo $\rightarrow$ \textbf{\#tempo} Duracion \num\\
		DefCompas $\rightarrow$ \textbf{\#compas} \num\textbf{/}\num\\
		Constantes $\rightarrow$ $\lambda$ $|$ constante \textbf{;} constantes\\
		Constante $\rightarrow$ \textbf{const} \var \textbf{ =} \num\\
		Voces $\rightarrow$ Voz \textbf{\}} $|$ Voz \textbf{\}} Voces\\
		Voz $\rightarrow$ \textbf{Voz (}Var\textbf{) \}} ListaCompases\\
		ListaCompases $\rightarrow$ ListaCompases CompORepe\\
		CopmORepe $\rightarrow$ Compases $|$ Repetir\\
		Repetir $\rightarrow$ \textbf{repetir (}\num \textbf{) \{} Compases \textbf{\}}\\
		Compases $\rightarrow$ Compas \textbf{\}} $|$ Compas \textbf{\}} Compases\\
		Compas $\rightarrow$ \textbf{compas \{} Figuras\\
		Figuras $\rightarrow$ Figura \textbf{;} $|$ Figura \textbf{;} Figuras\\
		Figura $\rightarrow$ Nota $|$ Silencio\\
		Nota $\rightarrow$ \textbf{nota (}\textit{\textbf{altura}}\textbf{,} Var\textbf{,} \textit{\textbf{duracion}}\textbf{)}\\
		Silencio $\rightarrow$ \textbf{silencio (}\textit{\textbf{duracion}}\textbf{)}\\
		Var $\rightarrow$ \var $|$ \num
	\end{minipage}
}

$\\$
Los tokens con valor son:\\

\fbox{
	\begin{minipage}{33em}
		\textit{\textbf{num}} = 0$|$[1-9][0-9]*\\
		\textit{\textbf{duracion}} = (redonda$|$blanca$|$negra$|$corchea$|$semicorchea$|$fusa$|$semifusa)(.)?\\
		\textit{\textbf{altura}} = (do$|$re$|$mi$|$fa$|$sol$|$la$|$si)(-$|$+)?\\
		\textit{\textbf{constante}} = [a-zA-Z]+
	\end{minipage}
}

$\\$

La gramática que se implementa en \texttt{parser\_rules.py} tiene una pequeña diferencia con
esta: no existe el no terminal ``ComoORepe'', que se agregó acá para simplificar la lectura, pero
está implementado directamente en la lista de compases.

\section*{Implementación}
\subsection*{Estructura}
El lenguaje se representa con una combinación de objetos y listas de estos objetos.  Como se verá
más adelante, estos objetos tienen métodos para la validación semántica del archivo y para la
conversión a \emph{midicomp}.

Un archivo parseado queda en un objeto \texttt{Musileng} que es una tupla \texttt{(DefTempo,
DefCompas, [Constantes] [Voces])}, y que se crea al reconocer la primer producción.
\texttt{DefTempo} y \texttt{DefCompas} tienen la información correspondiente a las siguientes dos
producciones; las constantes se toman como una lista de \texttt{(nombre, valor)} y se convierten en
un diccionario en el momento de construír el \texttt{Musileng}; y finalmente viene la lista de voces
con el resto de la información.

Una voz, representada en la clase \texttt{Voz}, consiste en la identificación del instrumento y una
lista de compases.  Los \texttt{repetir (N)} se implementan en el momento del parseo (en
\texttt{parser\_rules.py} en vez de \texttt{expressions.py}) generando N veces la lista interna de
compases.  Cada \texttt{Compas} consiste en una lista de objetos \texttt{Figura}, que pueden ser
de la subclase \texttt{Nota} o \texttt{Silencio}.

Para definir las figuras (y también para las definiciones de los compases y del \emph{tempo}) hay dos
clases: \texttt{Duracion} y \texttt{Altura}, que representan respectivamente el ritmo (redonda,
blanca, negra, con o sin puntillo, etc.) y la altura (do, re, mi, sostenido, bemol, etc.) de cada
figura.  Obviamente los silencios sólo tienen duración.

\subsection*{Validación}
La validación de la ``partitura'' se hace en forma \emph{top-down} llamando a una colección de
métodos \texttt{validar} que levantan una excepción con un mensaje (bastante) descriptivo cuando
detectan un error.  En la clase \texttt{Musileng}, se llama al \texttt{validar()} de cada una de las
voces, pasándoles la definición del compás esperado\footnotemark[1], más un número de voz, usado
para los mensajes de error.  Este método hace el pasamanos de esta información, más un número de
compás, también para los mensajes de error.

El \texttt{validar()} de cada compás hace la suma de las duraciones de sus figuras y la compara con
la duración esperada por la definición de compás.  Cada objeto \texttt{Duracion} ``sabe'' cuál es su
duración numérica: 1 para las redondas, 1/2 para las blancas, 1/4 para las negras, etc.
multiplicado por 1.5 en caso de haber puntillo.  El \texttt{Compas} levana una excepción si la suma
es mayor o menor a lo esperado.

\footnotetext[1]{Implementamos la verificación de que todas las voces tengan la misma cantidad de
	compases, pero dado que uno de los ejemplos de la cátedra viola esta regla, la verificación
	quedó comentada}

\subsection*{Conversión a formato \emph{midicomp}}
Una vez más, la implementación de esta tarea queda repartida en diversas clases, en las funciones
\texttt{get\_midicomp()}.  La clase \texttt{Musileng} lo único que sabe hacer es escribir el header,
pidiéndole la definición del \emph{tempo} en clicks a \texttt{DefTempo}, que implementa la fórmula
del enunciado.  Luego se van anexando los resultados del \texttt{get\_midicomp()} de cada voz.

Los objetos \texttt{Voz} reciben su id (un número asignado secuencialmente, salteando el
10\footnotemark[2]), los clicks por redonda, calculados según el enunciado y los pulsos por compás.
De esta información, lo único que usa \texttt{Voz} es el id, para escribir el header.  Los demás
datos, los pasa hacia cada uno de sus compases, junto con un nro. de compás, también generado
secuencialmente.  Cada compás genera su midicomp y este se va anexando al de la voz.

\footnotetext[2]{Con respecto a la percusión sólo tuvimos este cuidado de no generar una pista 10.
	No hay implementado nada especial para manejar voces de percusión ni presentamos ejemplos con
	percusión}

La clase \texttt{Compas} debe mantener, además del \emph{string} con el texto generado, el pulso y
click actual dentro del compás.  Para esto la clase \texttt{Figura} recibe en su
\texttt{get\_midicomp()} el pulso y click actual, además de los clicks por redonda y pulsos por
compás definidos ``arriba'' en \texttt{Musileng}.  A partir de estos datos, cada \texttt{Figura},
sabe dónde empieza y puede calcular dónde termina: primero se calcula el total de clicks que ocupará
la figura en función de su duración y los clicks por redonda; a partir de este valor, el pulso donde
termina la nota es 

$$
pulso_{final} = pulso_{inicial} + \lceil clicks_{total} / clicks\_por\_pulso \rceil
$$
\noindent
y el click es 
$$
clicks_{final} = clicks_{total} \% clicks\_por\_pulso
$$ 
\noindent
donde $clicks\_por\_pulso = 384$.

Se hace la excepción de que si $pulso_{final}$ = $pulsos\_por\_compas$, se escribe que la figura
finaliza en el 00:000 del siguiente compás.  El midicomp de cada figura se completa con la altura
correspondiente a cada nota y el volúmen, 70 en el caso de las notas y 0 en el de los silencios.  En
una primer versión de esta conversión los objetos \texttt{Silencio} solamente avanzaban pulsos y
clicks, generando un \texttt{string} vacío, pero fue necesario cambiar esto para para la
implementación de uno de los ejemplos, como se verá dos secciones más adelante.

\section*{Tests y ejemplos}
El \emph{testing} del programa es a tres niveles: para el \emph{lexer} usamos un \emph{unit test} que
verifica que cada token se reconozca correctamente.  En este nivel no hubo demasiadas
dificultades, tan solo algunos temas de precedencia que llevaban a que, por ejemplo, ``redonda''
fuera interpretado como ``Altura: re, Altura: do, Constante: nda'' en vez de como ``Duracion:
redonda''.

Para el parser encontramos que era muy difícil aislar cada posible producción, por lo que recurrimos
a tests del sistema completo, es decir ejemplos. Estos están en el directorio \texttt{files}, y son
los cuatro ejemplos de la cátedra, doce ejemplos de archivos con errores y tres ejemplos correctos.
Todos los archivos siguen el formato de los de la cátedra: \texttt{<titulo>\_input.txt}.

Los ejemplos erróneos están nombrados \texttt{erroneo\_\$i} con \texttt{\$i} del 1 al 12 y son:

\begin{enumerate}
	\item Constante definida dos veces
	\item Repetir vacío
	\item Repetir con N = 0
	\item Sin definición de \emph{tempo}
	\item Sin definición de compas
	\item Con una constante indefinida
	\item Sin voces
	\item Con una voz vacía
	\item Con un compás vacío
	\item Falta un ; en una figura
	\item Un compás es demasiado corto
	\item Un compás es demasiado largo
\end{enumerate}

Todos producen un error a nivel sintáctico o semántico (en los métodos \texttt{validar()}). El
mensaje que corresponde a los primeros intenta informar las posibles causas semánticas de la falla
que reporta el parser.

En cuanto a los ejemplos correctos, hay que empezar por decir que nuestro conocimiento de escritura
musical es limitado, apenas superior a lo explicado en el enunciado: Unos rudimentarios
conocimientos de rítmica, la capacidad de saber qué altura es cuál si contamos las partituras y
renglones y una ligera sospecha de algunos otros elementos.  Con esas armas nos arriesgamos a
implementar algunos ejemplos a partir de partituras buscadas en internet.

La elección del primer ejemplo a implementar nos resultó tan obvia a nosotros como a quienes nos
conocen: el solo de \emph{Ji Ji Ji}, de Patricio Rey y sus Redonditos de Ricota, que todos atribuímos al
saxo, y sin embargo en el disco está grabado con la guitarra.  Tuvimos que adaptar y corregir la
partitura que encontramos, hasta lograr que sonara más o menos parecida a lo que escuchábamos en el
disco.  El ejemplo aparece titulado \texttt{jijiji}

El sigiente ejemplo es un fragmento de otro clásico de nuestro rock: el \emph{riff} de
\emph{Post Crucifixión} de Pescado Rabioso.  Aquí nos encontramos con una limitación de
\emph{Musileng}, ya que el cierre del fragmento contiene dos \emph{síncopas}: notas que empiezan en
un compás y terminan en otro, y esto en el lenguaje no se puede implementar.  Lo emparchamos
definiendo los compases como $\frac{12}{8}$ en vez de la correcta $\frac{6}{8}$, con lo que una de las
síncopas queda dentro del compás, aunque la última resulta insalvable.  Este ejemplo se titula
\texttt{post\_crucifixion\_riff}.

Por último se adaptó sin demasiadas dificultades la pieza \emph{4:33} del compositor
contemporáneo John Cage.  Fue para lograr que este ejemplo quede correcto que se modificó la
conversión de los \texttt{Silencio} a \texttt{midicomp} para que escriba todos los tiempos, pues de
otra manera la pieza no tenía la duración correcta.

\section*{Uso del programa}
El ejecutable principal es el script \texttt{musileng}, que implementa la interfaz definida en el
enunciado.  Para simplificar el testeo implementamos un script \texttt{hacer\_midi.sh} que recibe el
título de un ejemplo, intenta generar el \texttt{midicomp} y, si lo logra, genera el midi y lo deja
en el directorio \texttt{midi}, como \texttt{<titulo>.mid}.  Opcionalmente puede pasársele un
\texttt{-p} para reproducir el midi generado.  Dentro del script se define qué programa se usa para
cada uno de estos pasos, en particular, usamos \emph{wildmidi} para la reproducción.

Hay un script extra, \texttt{hacer\_todo.sh} que ejecuta el script anterior para todos los ejemplos
definidos, escribiendo en pantalla, para los ejemplos erróneos, qué error se estaría probando.

\end{document}
