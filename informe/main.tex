\documentclass{article}
\usepackage[utf8]{inputenc}

\title{Informe TL}
\date{Julio 2015}

\usepackage{natbib}
\usepackage{graphicx}
\usepackage{caratula}
\usepackage[spanish]{babel}

\begin{document}

\newcommand{\num}{\textit{\textbf{num}}}
\newcommand{\var}{\textit{\textbf{constante}}}


%%%%%%%%%%%%%%%%%%%%%%%%%%%
%			INICIO DE CARÁTULA			%
%%%%%%%%%%%%%%%%%%%%%%%%%%%

%% **************************************************************************
%
%  Package 'caratula', version 0.2 (para componer caratulas de TPs del DC).
%
%  En caso de dudas, problemas o sugerencias sobre este package escribir a
%  Nico Rosner (nrosner arroba dc.uba.ar).
%
% **************************************************************************



% ----- Informacion sobre el package para el sistema -----------------------

\NeedsTeXFormat{LaTeX2e}
\ProvidesPackage{caratula}[2003/4/13 v0.1 Para componer caratulas de TPs del DC]


% ----- Imprimir un mensajito al procesar un .tex que use este package -----

\typeout{Cargando package 'caratula' v0.2 (21/4/2003)}


% ----- Algunas variables --------------------------------------------------

\let\Materia\relax
\let\Submateria\relax
\let\Titulo\relax
\let\Subtitulo\relax
\let\Grupo\relax


% ----- Comandos para que el usuario defina las variables ------------------

\def\materia#1{\def\Materia{#1}}
\def\submateria#1{\def\Submateria{#1}}
\def\titulo#1{\def\Titulo{#1}}
\def\subtitulo#1{\def\Subtitulo{#1}}
\def\grupo#1{\def\Grupo{#1}}


% ----- Token list para los integrantes ------------------------------------

\newtoks\intlist\intlist={}


% ----- Comando para que el usuario agregue integrantes

\def\integrante#1#2#3{\intlist=\expandafter{\the\intlist
	\rule{0pt}{1.2em}#1&#2&\tt #3\\[0.2em]}}


% ----- Macro para generar la tabla de integrantes -------------------------

\def\tablaints{%
	\begin{tabular}{|l@{\hspace{4ex}}c@{\hspace{4ex}}l|}
		\hline
		\rule{0pt}{1.2em}Integrante & LU & Correo electr\'onico\\[0.2em]
		\hline
		\the\intlist
		\hline
	\end{tabular}}


% ----- Codigo para manejo de errores --------------------------------------

\def\se{\let\ifsetuperror\iftrue}
\def\ifsetuperror{%
	\let\ifsetuperror\iffalse
	\ifx\Materia\relax\se\errhelp={Te olvidaste de proveer una \materia{}.}\fi
	\ifx\Titulo\relax\se\errhelp={Te olvidaste de proveer un \titulo{}.}\fi
	\edef\mlist{\the\intlist}\ifx\mlist\empty\se%
	\errhelp={Tenes que proveer al menos un \integrante{nombre}{lu}{email}.}\fi
	\expandafter\ifsetuperror}


% ----- Reemplazamos el comando \maketitle de LaTeX con el nuestro ---------

\def\maketitle{%
	\ifsetuperror\errmessage{Faltan datos de la caratula! Ingresar 'h' para mas informacion.}\fi
	\thispagestyle{empty}
	\begin{center}
	\vspace*{\stretch{2}}
	{\LARGE\textbf{\Materia}}\\[1em]
	\ifx\Submateria\relax\else{\Large \Submateria}\\[0.5em]\fi
	\par\vspace{\stretch{1}}
	{\large Departamento de Computaci\'on}\\[0.5em]
	{\large Facultad de Ciencias Exactas y Naturales}\\[0.5em]
	{\large Universidad de Buenos Aires}
	\par\vspace{\stretch{3}}
	{\Large \textbf{\Titulo}}\\[0.8em]
	{\Large \Subtitulo}
	\par\vspace{\stretch{3}}
	\ifx\Grupo\relax\else\textbf{\Grupo}\par\bigskip\fi
	\tablaints
	\end{center}
	\vspace*{\stretch{3}}
	\newpage}




\materia{Teor\'ia de Lenguajes}
\submateria{Primer Cuatrimestre de 2015}
\titulo{Trabajo Práctico 2}

\grupo{Grupo Estado Final}
\integrante{Gorojovsky, Román}{530/02}{rgorojovsky@gmail.com}
\integrante{Lazzaro, Leonardo}{147/05}{lazzaroleonardo@gmail.com}

\begin{titlepage}
\maketitle
\thispagestyle{empty}
\end{titlepage} 

%%%%%%%%%%%%%%%%%%%%%%%%%%%
%				FIN DE CARÁTULA			%
%%%%%%%%%%%%%%%%%%%%%%%%%%%

\section*{Introducción}
El trabajo consiste en implementar un programa que tome un archivo escrito en el lenguaje Musilen,
definido por la cátedra, y convertirlo en un archivo midi, pasando por un lenguaje intermedio que es
interpretado por el programa \emph{midicomp} para generar finalmente el archivo midi.

El problema, entonces puede dividirse en tres subproblemas: 

\begin{itemize}
	\item Convertir la descripción del lenguaje Musilen en una gramática bien definida.
	\item Aprender a usar alguna biblioteca preexistente para convertir esa gramática en código
	\item Convertir la salida del \emph{parser} creado en los dos pasos anteriores en el lenguaje de
		\emph{midicomp}
\end{itemize}

\section*{Decisiones tomadas}
Se eligió usar \emph{PLY} debido a nuestro mejor manejo del lenguaje \emph{Python} y se usó como
esqueleto del trabajo el código presentado en clase:

\begin{itemize}
	\item \texttt{lexer\_rules.py} contiene las definiciones y reglas de tokens
	\item \texttt{lexer.py} es el código de ejecución del \emph{lexer} (usado principalmente para
		testeo.
	\item \texttt{parser\_rules} contiene la definición de la gramática
	\item \texttt{expressions.py} contiene los objetos que se crean a partir del análisis sintáctico.
	\item \texttt{parser.py} es el código de ejecución del \emph{parser}
\end{itemize}

Como se verá en la sección que detalla la implementación, la conversión de objetos al lenguaje
\emph{midicomp} está implementada dentro de los objetos de \texttt{expressions.py}, usando patrones
de programación orientada a objetos.

\pagebreak
\section*{Gramática}
Se define la siguiente gramática, para cuya definición se priorizó la simplicidad en la construcción
de los objetos que luego se usarán para generar el archivo \emph{midicomp} por sobre la legibilidad
de la gramática. En general se pusieron los terminales (puntos y coma, llaves de cierre) en las
producciones ``de más afuera''.

Notamos en negrita los tokens y en negrita bastardilla los tokens con algún valor,
definidos más abajo.
$\\$

\fbox{
	\begin{minipage}{33em}
		Musilen $\rightarrow$ DefTempo DefCompas Constantes Voces\\
		DefTempo $\rightarrow$ \textbf{\#tempo} Duracion \num\\
		DefCompas $\rightarrow$ \textbf{\#compas} \num\textbf{/}\num\\
		Constantes $\rightarrow$ $\lambda$ $|$ constante \textbf{;} constantes\\
		Constante $\rightarrow$ \textbf{const} \var \textbf{ =} \num\\
		Voces $\rightarrow$ Voz \textbf{\}} $|$ Voz \textbf{\}} Voces\\
		Voz $\rightarrow$ \textbf{Voz (}Var\textbf{) \}} ListaCompases\\
		ListaCompases $\rightarrow$ ListaCompases CompORepe\\
		CopmORepe $\rightarrow$ Compases $|$ Repetir\\
		Repetir $\rightarrow$ \textbf{repetir (}\num \textbf{) \{} Compases \textbf{\}}\\
		Compases $\rightarrow$ Compas \textbf{\}} $|$ Compas \textbf{\}} Compases\\
		Compas $\rightarrow$ \textbf{compas \{} Figuras\\
		Figuras $\rightarrow$ Figura \textbf{;} $|$ Figura \textbf{;} Figuras\\
		Figura $\rightarrow$ Nota $|$ Silencio\\
		Nota $\rightarrow$ \textbf{nota (}\textit{\textbf{altura}}\textbf{,} Var\textbf{,} \textit{\textbf{duracion}}\textbf{)}\\
		Silencio $\rightarrow$ \textbf{silencio (}\textit{\textbf{duracion}}\textbf{)}\\
		Var $\rightarrow$ \var $|$ \num
	\end{minipage}
}

$\\$
Los tokens con valor son:\\

\fbox{
	\begin{minipage}{33em}
		\textit{\textbf{num}} = 0$|$[1-9][0-9]*\\
		\textit{\textbf{duracion}} = (redonda$|$blanca$|$negra$|$corchea$|$semicorchea$|$fusa$|$semifusa)(.)?\\
		\textit{\textbf{altura}} = (do$|$re$|$mi$|$fa$|$sol$|$la$|$si)(-$|$+)?\\
		\textit{\textbf{constante}} = [a-zA-Z]+
	\end{minipage}
}

$\\$

La gramática que se implementa en \texttt{parser\_rules.py} tiene una pequeña diferencia con
esta: no existe el no terminal ``ComoORepe'', que se agregó acá para simplificar la lectura, pero
está implementado directamente en la lista de compases.

\section*{Implementación}
\section*{Uso del programa}
\section*{Tests y ejemplos}


\end{document}
