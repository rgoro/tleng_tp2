\documentclass{article}
\usepackage[utf8]{inputenc}

\title{Informe TL}
\date{Julio 2015}

\usepackage{natbib}
\usepackage{graphicx}
\usepackage{caratula}
\usepackage[spanish]{babel}

\begin{document}

\newcommand{\num}{\textit{\textbf{num}}}
\newcommand{\var}{\textit{\textbf{constante}}}


%%%%%%%%%%%%%%%%%%%%%%%%%%%
%			INICIO DE CARÁTULA			%
%%%%%%%%%%%%%%%%%%%%%%%%%%%

%\include{caratula}

\materia{Teor\'ia de Lenguajes}
\submateria{Primer Cuatrimestre de 2015}
\titulo{Trabajo Práctico 2}

\grupo{Grupo Estado Final}
\integrante{Gorojovsky, Román}{530/02}{rgorojovsky@gmail.com}
\integrante{Lazzaro, Leonardo}{147/05}{lazzaroleonardo@gmail.com}

\begin{titlepage}
\maketitle
\thispagestyle{empty}
\end{titlepage} 

%%%%%%%%%%%%%%%%%%%%%%%%%%%
%				FIN DE CARÁTULA			%
%%%%%%%%%%%%%%%%%%%%%%%%%%%

\section*{Introducción}
El trabajo consiste en implementar un programa que tome un archivo escrito en el lenguaje Musilen,
definido por la cátedra, y convertirlo en un archivo midi, pasando por un lenguaje intermedio que es
interpretado por el programa \emph{midicomp} para generar finalmente el archivo midi.

El problema, entonces puede dividirse en tres subproblemas: 

\begin{itemize}
	\item Convertir la descripción del lenguaje Musilen en una gramática bien definida.
	\item Aprender a usar alguna biblioteca preexistente para convertir esa gramática en código
	\item Convertir la salida del \emph{parser} creado en los dos pasos anteriores en el lenguaje de
		\emph{midicomp}
\end{itemize}

\section*{Decisiones tomadas}
Se eligió usar \emph{PLY} debido a nuestro mejor manejo del lenguaje \emph{Python} y se usó como
esqueleto del trabajo el código presentado en clase:

\begin{itemize}
	\item \texttt{lexer\_rules.py} contiene las definiciones y reglas de tokens
	\item \texttt{lexer.py} es el código de ejecución del \emph{lexer} (usado principalmente para
		testeo.
	\item \texttt{parser\_rules} contiene la definición de la gramática
	\item \texttt{expressions.py} contiene los objetos que se crean a partir del análisis sintáctico.
	\item \texttt{parser.py} es el código de ejecución del \emph{parser}
\end{itemize}

Como se verá en la sección que detalla la implementación, la conversión de objetos al lenguaje
\emph{midicomp} está implementada dentro de los objetos de \texttt{expressions.py}, usando patrones
de programación orientada a objetos.

\pagebreak
\section*{Gramática}
Se define la siguiente gramática, para cuya definición se priorizó la simplicidad en la construcción
de los objetos que luego se usarán para generar el archivo \emph{midicomp} por sobre la legibilidad
de la gramática. En general se pusieron los terminales (puntos y coma, llaves de cierre) en las
producciones ``de más afuera''.

Notamos en negrita los tokens y en negrita bastardilla los tokens con algún valor,
definidos más abajo.
$\\$

\fbox{
	\begin{minipage}{33em}
		Musilen $\rightarrow$ DefTempo DefCompas Constantes Voces\\
		DefTempo $\rightarrow$ \textbf{\#tempo} Duracion \num\\
		DefCompas $\rightarrow$ \textbf{\#compas} \num\textbf{/}\num\\
		Constantes $\rightarrow$ $\lambda$ $|$ constante \textbf{;} constantes\\
		Constante $\rightarrow$ \textbf{const} \var \textbf{ =} \num\\
		Voces $\rightarrow$ Voz \textbf{\}} $|$ Voz \textbf{\}} Voces\\
		Voz $\rightarrow$ \textbf{Voz (}Var\textbf{) \}} ListaCompases\\
		ListaCompases $\rightarrow$ ListaCompases CompORepe\\
		CopmORepe $\rightarrow$ Compases $|$ Repetir\\
		Repetir $\rightarrow$ \textbf{repetir (}\num \textbf{) \{} Compases \textbf{\}}\\
		Compases $\rightarrow$ Compas \textbf{\}} $|$ Compas \textbf{\}} Compases\\
		Compas $\rightarrow$ \textbf{compas \{} Figuras\\
		Figuras $\rightarrow$ Figura \textbf{;} $|$ Figura \textbf{;} Figuras\\
		Figura $\rightarrow$ Nota $|$ Silencio\\
		Nota $\rightarrow$ \textbf{nota (}\textit{\textbf{altura}}\textbf{,} Var\textbf{,} \textit{\textbf{duracion}}\textbf{)}\\
		Silencio $\rightarrow$ \textbf{silencio (}\textit{\textbf{duracion}}\textbf{)}\\
		Var $\rightarrow$ \var $|$ \num
	\end{minipage}
}

$\\$
Los tokens con valor son:\\

\fbox{
	\begin{minipage}{33em}
		\textit{\textbf{num}} = 0$|$[1-9][0-9]*\\
		\textit{\textbf{duracion}} = (redonda$|$blanca$|$negra$|$corchea$|$semicorchea$|$fusa$|$semifusa)(.)?\\
		\textit{\textbf{altura}} = (do$|$re$|$mi$|$fa$|$sol$|$la$|$si)(-$|$+)?\\
		\textit{\textbf{constante}} = [a-zA-Z]+
	\end{minipage}
}

$\\$

La gramática que se implementa en \texttt{parser\_rules.py} tiene una pequeña diferencia con
esta: no existe el no terminal ``ComoORepe'', que se agregó acá para simplificar la lectura, pero
está implementado directamente en la lista de compases.

\section*{Implementación}
\section*{Uso del programa}
\section*{Tests y ejemplos}


\end{document}
